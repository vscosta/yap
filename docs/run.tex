@c -*- mode: texinfo; coding: utf-8; -*-

@node Run, Syntax, Install, Top
@chapter Running YAP

@menu
* Running YAP Interactively:: Interacting with YAP
* Running Prolog Files:: Running Prolog files as scripts
@end menu

@cindex booting
We next describe how to invoke YAP in Unix systems.

@node Running YAP Interactively, ,Running Prolog Files,Run
@section Running YAP Interactively

Most often you will want to use YAP in interactive mode. Assuming that
YAP is in the user's search path, the top-level can be invoked under
Unix with the following command:

@example
yap [-s n] [-h n] [-a n] [-c IP_HOST port ] [filename]
@end example

@noindent
All the arguments and flags are optional and have the following meaning:
@table @code
@item -?
print a short error message.
@item -s@var{Size}
allocate @var{Size} KBytes for local and global stacks. The user may
 specify @t{M} bytes.
@item -h@var{Size}
allocate @var{Size} KBytes for heap and auxiliary stacks
@item -t@var{Size}
allocate @var{Size} KBytes for the trail stack
@item -L@var{Size} 
SWI-compatible option to allocate @var{Size} K bytes for local and global stacks, the local stack
 cannot be expanded. To avoid confusion with the load option, @var{Size}
 must immediately follow the letter @code{L}.
@item -G@var{Size}
SWI-compatible option to allocate @var{Size} K bytes for local and global stacks; the global
stack cannot be expanded
@item -T@var{Size}
SWI-compatible option to allocate @var{Size} K bytes for the trail stack; the trail cannot be expanded.
@item -l @var{YAP_FILE}
compile the Prolog file @var{YAP_FILE} before entering the top-level.
@item -L @var{YAP_FILE}
compile the Prolog file @var{YAP_FILE} and then halt. This option is
useful for implementing scripts.
@item -g @var{Goal}
run the goal @var{Goal} before top-level. The goal is converted from
an atom to a Prolog term.
@item -z @var{Goal}
run the goal @var{Goal} as top-level. The goal is converted from
an atom to a Prolog term.
@item -b @var{BOOT_FILE}
boot code is in Prolog file @var{BOOT_FILE}. The filename must define
the predicate @code{'$live'/0}.
@item -c @t{IP_HOST} @t{port}
connect standard streams to host @t{IP_HOST} at port @t{port}
@item filename
restore state saved in the given file
@item -f
do not consult initial files
@item -q
do not print informational messages
@item --
separator for arguments to Prolog code. These arguments are visible
through the @code{unix/1} built-in predicate.
@end table

Note that YAP will output an error message on the following conditions:

@itemize @bullet
@item
a file name was given but the file does not exist or is not a saved
YAP state;
@item 
the necessary amount of memory could not be allocated;
@item
the allocated memory is not enough to restore the state.
@end itemize

When restoring a saved state, YAP will allocate the
same amount of memory as that in use when the state was saved, unless a
different amount is specified by flags in the command line. By default,
YAP restores the file @file{startup.yss} from the current directory or from
the YAP library.
@cindex environment variables

@findex YAPBINDIR
@itemize @bullet
@item
YAP usually boots from a saved state. The saved state will use the default
installation directory to search for the YAP binary unless you define
the environment variable YAPBINDIR.

@findex YAPLIBDIR
@item
YAP always tries to find saved states from the current directory
first. If it cannot it will use the environment variable YAPLIBDIR, if
defined, or search the default library directory.

@findex YAPSHAREDIR
@item
YAP will try to find library files from the YAPSHAREDIR/library
directory.
@end itemize

@node Running Prolog Files, Running YAP Interactively, , Run
@section Running Prolog Files

YAP can also be used to run Prolog files as scripts, at least in
Unix-like environments. A simple example is shown next (do not forget
that the shell comments are very important):

@example
@cartouche
#!/usr/local/bin/yap -L --
#
# Hello World script file using YAP
#
# put a dot because of syntax errors .

:- write('Hello World'), nl.

@end cartouche
@end example

The @code{#!}  characters specify that the script should call the binary
file YAP. Notice that many systems will require the complete path to the
YAP binary. The @code{-L} flag indicates that YAP should consult the
current file when booting and then halt. The remaining arguments are
then passed to YAP. Note that YAP will skip the first lines if they
start with @code{#} (the comment sign for Unix's shell). YAP will
consult the file and execute any commands.

A slightly more sophisticated example is:

@example
@cartouche
#!/usr/bin/yap -L --
#
# Hello World script file using YAP
# .

:- initialization(main).

main :- write('Hello World'), nl.

@end cartouche
@end example

The @code{initialization} directive tells YAP to execute the goal main
after consulting the file. Source code is thus compiled and @code{main}
executed at the end. The @code{.} is useful while debugging the script
as a Prolog program: it guarantees that the syntax error will not
propagate to the Prolog code.

Notice that the @code{--} is required so that the shell passes the extra
arguments to YAP.  As an example, consider the following script
@code{dump_args}:

@example
@cartouche
#!/usr/bin/yap -L --
#.

main( [] ).
main( [H|T] ) :-
        write( H ), nl,
        main( T ).

:- unix( argv(AllArgs) ), main( AllArgs ).

@end cartouche
@end example

If you this run this script with the arguments:
@example
./dump_args -s 10000
@end example
@noindent
the script will start an YAP process with stack size @code{10MB}, and
the list of arguments to the process will be empty.

Often one wants to run the script as any other program, and for this it
is convenient to ignore arguments to YAP. This is possible by using
@code{L --} as in the next version of @code{dump_args}:

@example
@cartouche
#!/usr/bin/yap -L --

main( [] ).
main( [H|T] ) :-
        write( H ), nl,
        main( T ).

:- unix( argv(AllArgs) ), main( AllArgs ).

@end cartouche
@end example

The @code{--} indicates the next arguments are not for YAP. Instead,
they must be sent directly to the @code{argv} built-in. Hence, running
@example
./dump_args test
@end example
@noindent
will write @code{test} on the standard output.


